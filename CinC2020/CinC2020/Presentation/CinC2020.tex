\documentclass{beamer}

\bibliographystyle{Latex/Classes/PhDbiblio-case}

\usepackage[latin1]{inputenc}
\usepackage{graphicx}
\usepackage{amsmath,amssymb,amsthm}
\usepackage[mathscr]{eucal}
\usepackage{textcomp}
\usepackage{subfig}
\usepackage{epsfig}
\usepackage{multirow}
\usepackage{hhline}
\usepackage{bm}

\beamertemplatetextbibitems

\usetheme{Boadilla}
\usecolortheme{seahorse}
\usepackage[latin1]{inputenc}

\title[CinC 2020]{Tensor-Based Noninvasive Atrial Fibrillation Complexity Index For Catheter Ablation} 
\author[Lucas de S. Abdalah]{Lucas de S. Abdalah, Pedro Marinho R. de Oliveira, Vicente Zarzoso, Walter Freitas Jr.}
\date{Sep $16^{th}$, 2020} 

\titlegraphic{\includegraphics[scale = 0.055]{UCA_logo} \includegraphics[scale = 0.150]{I3S_logo}}

\begin{document}

\frame{\titlepage}

\section[Outline]{}
\frame{\tableofcontents}


\section{Introduction} 

\frame {

	\begin{center}
		\Large{Introduction}
	\end{center}

}

\frame {
	\frametitle{Atrial Fibrillation}
	
	\begin{itemize}
		\item Atrial Fibrillation (AF) is the most common sustained cardiac arrhythmia in clinical practice.
	\end{itemize}
	
	\begin{figure}
		\centering
		\includegraphics[scale=0.2]{AF_ECG}
	\end{figure}
	\vspace{-0.8in}
	\begin{itemize}
		\item The mechanisms behind AF are complex and not completely understood
		\begin{itemize}
			\item Intensive clinical research increased in the past few years.
		\end{itemize}
	\end{itemize}
	
}

\frame {
	\frametitle{Signal Processing Techniques}
	
	\begin{itemize}
		\item Signal processing techniques appear as a key tool to understand and manage better this challenge.
	\end{itemize}
	
	\vspace{0.3in}
	
	The electrocardiogram (ECG) data matrix can be modeled as:
	\begin{equation}
	\textbf{Y} = \textbf{MS} \in \mathbb{R}^{K \times N} \; ,
	\label{bssp}
	\end{equation}
	where $\textbf{M} \in \mathbb{R}^{K \times R}$ is a mixing matrix and $\textbf{S} \in \mathbb{R}^{R \times N}$ is the source matrix.	
	
	\vspace{0.3in}
	
	\begin{itemize}
		\item Blind source separation (BSS) problem
		\begin{itemize}
			\item Goal: to estimate $\textbf{M}$ and $\textbf{S}$ knowing only $\textbf{Y}$.
		\end{itemize}
	\end{itemize}
	
}
	
\frame {
	\frametitle{Atrial Source Classification}
	
	\begin{block}{Challenge}
		After performing the BSS, the automated atrial activity (AA) source classification is still a problem.
	\end{block}

	\begin{itemize}
		\item The spectral concentration (SC) was proposed as an AA quality index\footnote{Castells \textit{et al}., ``Spatiotemporal blind source separation approach to atrial activity estimation in atrial tachyarrhythmias'', \textit{IEEE Trans. Biomed. Eng.}, 2005.}.
			\item Choose the source with highest SC
			\begin{itemize}
				\item Weak sources can be mistaken as atrial sources.
			\end{itemize}
\end{itemize}

}

\frame {
	\frametitle{Atrial Source Classification}
	
		\begin{itemize}
			\item The kurtosis in the frequency domain was introduced as a new AA quality index\footnote{\label{LVAICA}De Oliveira and Zarzoso, ``Source analysis and selection using block term decomposition in atrial fibrillation'', in \textit{Proc. LVA/ICA}, 2018.}.
		\end{itemize}
		
	\begin{itemize}
		\item Two new methods of AA source selection were proposed\footnotemark[2]
		\begin{enumerate}
			\item Discard weak sources.
			\item Select the source with higher SC (PM1) or higher kurtosis (PM2).
		\end{enumerate}
	\end{itemize}
	\vspace{0.2in}
	\begin{center}
		{\Large Accuracy can still be improved!}
	\end{center}
	\vspace{0.2in}
	\begin{itemize}
		\item A better index to measure AA estimation quality.
		\item Machine learning algorithms for classification.
	\end{itemize}
	
}

\section{Methods}

\frame {

	\begin{center}
		\Large{Methods}
	\end{center}

}
	
\frame { 
	\frametitle{Blind Source Separation Techniques} 
	
	\begin{itemize}
		\item Principal component analysis (PCA)
		\begin{itemize}
			\item Orthogonal linear transformation in the original data.
			\item Sources are mutually uncorrelated.
		\end{itemize}
		\vspace{0.1in}
		\item Robust independent component analysis (RobustICA-f)
		\begin{itemize}
			\item Variant of ICA.
			\item Statistically independent sources.
		\end{itemize}
		\vspace{0.1in}
		\item Hankel-based block term decomposition (BTD)
		\begin{itemize}
			\item Tensor factorization technique.
			\item Explores the nature of AA during AF.
		\end{itemize}
	\end{itemize}
		
}

\frame { 
	\frametitle{Machine Learning Algorithms}
	
	\begin{block}{Features}
		\begin{itemize}
			\item SC.
			\item Kurtosis.
			\item The normalized mean square error of the TQ segment (NMSE-TQ) on lead V1 between the original recording and the source estimate
				\begin{equation}
					NMSE\textrm{-}TQ = \left[ \frac{||m^{(V1)}_{r} \textbf{s}_{r.} - \textbf{y}^{(V1)}||_F^2}{||\textbf{y}^{(V1)}||_F^2} \right]_{TQ} \; .
				\end{equation}
		\end{itemize}
	\end{block}
	
	\begin{itemize}
		\item Classifiers
		\begin{itemize}
			\item Linear and quadratic discriminant analysis (LDA and QDA).
			\item Support vector machine (SVM).
		\end{itemize}
	\end{itemize}
	
	
} 

\section{Experimental Results}

\frame {

	\begin{center}
		\Large{Experimental Results}
	\end{center}

}

\frame { 
	\frametitle{Database and Experimental Setup} 
	
	\begin{block}{Database}
		\begin{itemize}
			\item 30 patients suffering from persistent AF.
			\item ECG segments from 0.82 to 1.75 seconds.
		\end{itemize}
		
		\begin{center}
			Cardiology Department of Princess Grace Hospital Center, Monaco.
		\end{center}					
	\end{block}	
	
	\begin{itemize}
		\item RobustICA-f was implemented using the RobustICA toolbox.
		\item Hankel-based BTD was implemented using the Tensorlab toolbox.
	\end{itemize}

}

\frame { 
	\frametitle{Classification} 
	
		\begin{columns}
		\begin{column}{0.5\textwidth}
			\begin{figure}[t]
				\vspace{-0.3in}
				%\centering
				\includegraphics[scale=0.2]{Figure1.eps}
				%\caption{3. Distribution of SC values (\%) for each version of BTD, as well as PCA and RobustICA-f.}
				\label{Figure2}
			\end{figure}
		\end{column}
		\begin{column}{0.5\textwidth}  %%<--- here
   			\begin{itemize}
   				\item 1283 sources
   				\begin{itemize}
   					\item 551 AA sources.
   					\item 732 non-AA sources.
   				\end{itemize}
   				\item Lead V1 is taken as reference.
   				\item 80\% for training and 20\% for testing.
   			\end{itemize}
		\end{column}
	\end{columns}

}

\frame {
	\frametitle{Classification}
	
	\begin{figure}[t]
		\vspace{-0.3in}
		%\centering
		\includegraphics[scale=0.38]{Figure2.eps}
		%\caption{3. Distribution of SC values (\%) for each version of BTD, as well as PCA and RobustICA-f.}
		\label{Figure2}
	\end{figure}

}

\frame {
	\frametitle{Classification}
	
	\begin{block}{SC ($\mu \pm \sigma$)}
		\begin{itemize}
			\item AA sources: $61.8 \pm 15.1$
			\item non-AA sources: $37.8 \pm 14.5$
		\end{itemize}
	\end{block}
	
	\begin{block}{Kurtosis ($\mu \pm \sigma$)}
		\begin{itemize}
			\item AA sources: $141.3 \pm 65.4$
			\item non-AA sources: $41.7 \pm 29.1$
		\end{itemize}
	\end{block}
	
	
	\begin{block}{NMSE-TQ ($\mu \pm \sigma$)}
		\begin{itemize}
			\item AA sources: $1.6 \pm 3.0$
			\item non-AA sources: $34.9 \pm 111.0$
		\end{itemize}
	\end{block}


}

\section{Conclusions} 

\frame {

	\begin{center}
		\Large{Conclusions}
	\end{center}
	
	
}

\frame {
	\frametitle{Conclusions}
	
	\begin{itemize}
		\item BSS methods are important tools for AA extraction in AF ECGs
		\begin{itemize}
			\item AA estimation quality is challenging without a ground-truth.
			\item Automated source classification can still improve.
		\end{itemize}
	\end{itemize}
	
	\begin{block}{Contributions}
		\begin{itemize}
			\item A new index for AA estimation quality measurement.
			\item AA source classification with machine learning algorithms.
			\item Assessment in a database of 30 patients.
		\end{itemize}
	\end{block}
	
	\begin{block}{Future Works}
		\begin{itemize}
			\item Assess the performance of other classifiers.
			\item Perform unsupervised classification.
			\item Increase the database of patients.
		\end{itemize}
	\end{block}
	
}

\end{document}
